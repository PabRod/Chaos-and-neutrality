\section{Background}
\label{sec:Background} 
Ever since Darwin, the idea that species must be sufficiently different to be able to coexist is deeply rooted in the history of biological thinking. Indeed, the principle of competitive exclusion is intuitively straightforward, and elegant mathematical underpinning\cite{MacArthur} helped making it one of the cornerstones of ecological theory. Nevertheless, on a closer examination, natural communities often seem to harbor far more species that may be reasonably explained from niche separation. Plankton communities, where many species coexist with little room for differentiation, have served as an early example \cite{Hutchinson, Hutchinson1961}, inspiring the legendary ecologist G. Evelyn Hutchinson to ask the simple but fundamental question \textit{"why are there so many kinds of animals?"} \cite{Hutchinson1961}. Since then many mechanisms have been shown to help similar species coexist. As Hutchinson already proposed himself, fluctuations in conditions may prevent reaching equilibrium at which species would be outcompeted. Also, natural enemies including pests and parasites tend to attack the abundant species more than rare species, and such a \textit{"kill the winner"}\cite{Winter2010} mechanism promotes diversity by preventing one species to take all the resources and outcompete the rest.

In the extensive literature on mechanisms that can prevent competitive exclusion there are two newcomers that radically differ from the rest and have created quite a stir: neutrality and chaos. The neutral theory of biodiversity introduced by Hubbell \cite{Hubbell2001} proposes that species that are entirely equivalent can coexist in a neutral way because none is able to outcompete the other. The concept of completely equivalent species has met skepticism as it is incompatible with the idea that all species are different. However, it turns out that also \textit{"near-neutrality"} arises robustly in models of competition and evolution and may boost the chances for coexistence \cite{Scheffer2006, Scheffer2018, Fort2009, Fort2010}. Support for such near-neutrality has been found in a wide range of communities \cite{Scheffer2006, Vergnon2013, Scheffera, Segura2013, Vergnon2012}. The second relatively new and controversial mechanism that may prevent competitive exclusion is \textit{"super-saturated coexistence"} in communities that display chaotic dynamics \cite{Huisman1999}. This is in a sense analogous to the prevention of competitive exclusion in fluctuating environments, except that deterministic chaos may arise in autonomous non-linear systems without any external perturbation. Although there has been much debate about the question whether such internally driven complex dynamics plays an important role in ecosystems, several studies support the idea that chaos can be an essential ingredient of natural dynamics \cite{Beninca2008}.

Surprisingly, while the potential roles of chaos and neutrality have been intensely debated, no studies seem to have explored how these two fundamental drivers of diversity could be causally related. Here we address this question using simple food-web models. We vary the level of neutrality in the competition between prey species and analyze its effect on the likelihood of generating chaotic dynamics.