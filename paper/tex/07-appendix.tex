\appendix
\setcounter{equation}{0}
\setcounter{figure}{0}
\renewcommand{\theequation}{A.\arabic{equation}}
\renewcommand\thefigure{A.\arabic{figure}}

\section{Online appendix}
\label{sec:Appendix}

This is the Online Appendix for the paper:

\begin{center}
Rodríguez-Sánchez P, van Nes EH, Scheffer M. \textit{Neutral competition boosts chaos in food webs}.
\end{center}

\newpage

\subsection{Results for food webs of different sizes}
\label{subsec:GeneralResults}
In the main body of the paper we focused our attention in families of food webs consisting of $12$ prey and $8$ predator species. In this section we show the results of the same analysis for food webs of different sizes.

\subsubsection{Probability of chaos grouped by number of species}
\label{subsubsec:AllProbabilities}
\begin{figure}[H]
	\begin{center}
		\includegraphics[width=1\columnwidth]{results_all.png}
	\end{center}
	\caption{Probabilities of chaos vs. competition parameter for the whole set of simulations. The competition parameter $\epsilon$ is on the horizontal axis. The estimated probability of chaos is represented on the vertical one. Each panel corresponds to an ecosystem with a different number of interacting species. The exact number is shown in each box, as number of predator + number of prey species.}
	\label{fig:AllProbabilities}
\end{figure}

\subsubsection{Probability of each dynamical regime}
\label{subsubsec:DynamicalRegimes}
\begin{figure}[H]
	\begin{center}
		\includegraphics[width=1\columnwidth]{ratios.png}
	\end{center}
	\caption{Ratio of each dynamical regime vs. competition parameter for the whole set of simulations. The competition parameter $\epsilon$ is on the horizontal axis. The system size is shown in each box, as number of predator + number of prey species.}
	\label{fig:DynamicalRegimes}
\end{figure}

\subsubsection{Biodiversity measurements}
\label{subsubsec:BiodiversityFigs}

For each simulation, a biodiversity index was estimated as the number of prey species whose population was higher than a minimum threshold of $\bioThreshold$ $mg$ $l^{-1}$, averaged respective to time.

\begin{figure}[H]
	\begin{center}
		\includegraphics[width=1\columnwidth]{biod_split_by_dynamics.png}
	\end{center}
	\caption{Average prey biodiversity vs. competition parameter. Each panel shows a food network of a different size. For each value of the competition parameter, 200 randomly drawn ecosystems were simulated. The dashed line shows the average number of prey species of these 200 simulations. The yellow circles represent the average prey biodiversity of those simulations who had chaotic dynamics. The red and blue circles represent the same for, respectively, cyclic and stable dynamics. The relative area of the circles represents the ratio of each kind of dynamics.}
	\label{fig:BiodSplitByChaos}
\end{figure}

\begin{figure}[H]
	\begin{center}
		\includegraphics[width=1\columnwidth]{biod_box_and_whisker.png}
	\end{center}
	\caption{Box and whisker plot of the prey biodiversity, after being classified as stable, cyclic or chaotic.}
	\label{fig:BiodBoxAndWhisker}
\end{figure}

\begin{figure}[H]
	\begin{center}
		\includegraphics[width=1\columnwidth]{biomass.png}
	\end{center}
	\caption{Average biomasses grouped by trophic level vs. competition parameter. The width represents standard deviation.}
	\label{fig:Biomass}
\end{figure}
