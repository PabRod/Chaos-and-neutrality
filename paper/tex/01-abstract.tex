\begin{abstract}
\label{sec:Abstract}
Near-neutrality of competition has been proposed to facilitate coexistence of species because it slows down competitive exclusion, thus making it easier for equalizing mechanisms to maintain diverse communities. An unrelated line of work has shown that chaos can promote coexistence of many species in super-saturated communities. By analyzing a set of numerically simulated food webs, here we link those previously unrelated findings. We show that near-neutrality of competition at the prey's trophic level, in the presence of interactions with natural enemies, increases the chances of developing chaotic dynamics and we show that this results in a higher biodiversity. Our results suggest that near-neutrality may promote biodiversity in two ways: through reducing the rates of competitive displacement and through promoting non-equilibrium dynamics.

\end{abstract}
