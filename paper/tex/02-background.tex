\section{Background}
\label{sec:Background}
Ever since Darwin, the idea that species must be sufficiently different to coexist is deeply rooted in biological thinking. Indeed, the principles of limiting similarity \citep{MacArthur} and competitive exclusion \citep{Hardin1960, Armstrong1980} are the cornerstones of ecological theory. Nevertheless, natural communities (such as plankton communities \citep{Hutchinson1961}), often harbor far more species that may be explained from niche separation, inspiring G. Evelyn Hutchinson \citeyearpar{Hutchinson} to ask the simple but fundamental question \textit{"why are there so many kinds of animals?"}. Since then many mechanisms have been suggested that may help similar species to coexist. As Hutchinson \citeyearpar{Hutchinson1961} already proposed himself, fluctuations in conditions may prevent populations to reach equilibrium at which species would be outcompeted. Also, natural enemies including pests and parasites tend to attack the abundant species more than rare species, and such a \textit{"kill the winner"} \citep{Winter2010} mechanism promotes diversity by preventing one species to become dominant.

In the extensive literature on potential mechanisms that could prevent competitive exclusion there are two relatively new ideas that have created some controversy: neutrality and chaos. The neutral theory of biodiversity introduced by Hubbell \citeyearpar{Hubbell2001} proposes that species that are entirely equivalent can coexist because none is able to outcompete the other. The concept of equivalent species has met skepticism as it is incompatible with the idea that all species are different. However, it turns out that also \textit{"near-neutral"} competitors can coexist in models of competition and evolution \citep{Scheffer2006, Scheffer2018}. Support for such near-neutrality has been found in a wide range of communities \citep{Vergnon2013, Segura2013}. The second controversial mechanism that may prevent competitive exclusion is \textit{"super-saturated coexistence"} in communities that display chaotic dynamics \citep{Huisman1999}. This is in a sense analogous to the prevention of competitive exclusion in fluctuating environments, except that deterministic chaos is internally driven. Although there has been much debate about the question whether chaotic dynamics plays an important role in ecosystems \citep{Berryman1989, Scheffer1991, Schippers2001}, several studies support the idea that chaos can be an essential ingredient of natural dynamics \citep{Huisman1999, Beninca2008, Beninca}.

Intuitively, it seems not likely that chaos and neutrality can be related, as fully neutral ecosystems can't be chaotic. However, natural ecosystems are never perfectly neutral, and predators may have a preference for different species. In the present work, we used a multi-species food-web model to explore the effect of near-neutrality of prey on the probability of developing chaotic dynamics. We found a surprising link between both ideas: the closer to neutrality the competition is, the higher the chances of developing chaotic dynamics. Additionally, our results confirmed that there is a positive relation between cyclic or chaotic dynamics and the number of coexisiting species.
